\documentclass{article}

\usepackage{amsfonts,amsthm,amsmath,enumitem,multicol}
\usepackage[margin=.5in,bottom=1in]{geometry}

\newcommand{\boxtalk}[1]{
    \fbox{
    \begin{minipage}{\textwidth}
        #1
    \end{minipage}
    }
}

\setlength{\parindent}{0cm}


\newcommand{\makeheader}[3]{
    {
    \bfseries
    Name:\underline{\hspace*{3in}}\\\mbox{}\\
    #1\hfill Math 109\\
    #2\hfill #3\\
    \hrule\mbox{}\\
    }
}

\newcommand{\makeheadernext}[3]{
    {
    \bfseries
    #1\hfill Math 109\\
    #2\hfill #3\\
    \hrule\mbox{}\\
    }
}

\newcommand{\skills}[3]{
    \textbf{Skills Practice #1} (#2) #3
}

\newcommand{\alphcols}[2]{
    \begin{multicols}{#1}
    \begin{enumerate}[label=\textbf{\alph*.}]
        #2
    \end{enumerate}
    \end{multicols}  
    }

\begin{document}
\makeheader%
    {  Sections R.2-R.3  }% Section
    {  Review on Exponents }% Topic Summary
    { 9/8/23  }% Date


%%%
% Skills Practice
%%%

\skills{1}{R.2}{ Simplify
    \alphcols{3}{\item $4z^0$\item$(4z)^0$\item $-4^0$}
    \boxtalk{
        \textbf{Note:} \underline{Simplify} will mean to write the expression so that there are no $0$ exponents.
    }
    }
\vfill

\skills{2}{R.2}{ Simplify
    \alphcols{3}{\item $2^{-3}$\item$\dfrac{1}{n^{-5}}$\item $-3a^4b^{-9}$}
    \boxtalk{
        \textbf{Note:} \underline{Simplify} will mean to write the expression so that there are no negative exponents.
    }
    }
\vfill
    
\skills{1}{R.3}{ Simplify
    \alphcols{5}{\item $\sqrt[3]{-125}$\item $\sqrt{\dfrac{144}{121}}$\item $\sqrt[5]{0.00001}$\item $\sqrt[6]{-64}$\item $-\sqrt[6]{64}$}
    \boxtalk{
        \textbf{Note:} \underline{Simplify} will mean to take all perfect powers outside the root or state that the root is not a real number.
    }
    }
\vfill
        
\skills{2}{R.3}{ Simplify
    \alphcols{5}{\item $36^{1/2}$\item $\left(\dfrac{1}{125}\right)^{1/3}$\item $(-9)^{1/2}$\item $(-1)^{4/3}$\item $(16)^{3/4}$}
    \boxtalk{
        \textbf{Note:} \underline{Simplify} will mean to take all perfect powers outside the fractional exponent or state that the root is not a real number.
    }
}
\vfill

\newpage

\makeheadernext%
    {  Sections R.2-R.3  }% Section
    {  Review on Exponents }% Topic Summary
    { 9/8/23  }% Date
%%%%%%%
% In Class Exercises
%%%%%%%
\begin{enumerate}
    \item Simplify $\sqrt[3]{125}$.\vfill
    \item Simplify $8^{2/3}$.\vfill
    \item Simplify $\dfrac{y^{7/5}y^{4/5}}{y^{1/5}}$.\vfill
    \item Simplify $\sqrt{d^{11}}$. Here $d$ represents a positive real number.\vfill
    \item Simplify $\sqrt[3]{40ab^{13}{c^{17}}}$. Here $a,b$ and $c$ represent positive real numbers.\vfill
    \item Multiply $\sqrt{6}\sqrt{21}$.\vfill
\end{enumerate}
\end{document}